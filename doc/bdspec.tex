%% Copyright 2010 Richard Wood. All rights reserved.
%% 
%% Redistribution and use in source and binary forms, with or without
%% modification, are permitted provided that the following conditions are met:
%% 
%%    1. Redistributions of source code must retain the above copyright notice,
%%    this list of conditions and the following disclaimer.
%% 
%%    2. Redistributions in binary form must reproduce the above copyright
%%    notice, this list of conditions and the following disclaimer in the
%%    documentation and/or other materials provided with the distribution.
%% 
%% This software is provided by Richard Wood ``as is'' and any express or
%% implied warranties, including, but not limited to, the implied warranties of
%% merchantability and fitness for a particular purpose are disclaimed. In no
%% event shall Richard Wood or contributors be liable for any direct,
%% indirect, incidental, special, exemplary, or consequential damages
%% (including, but not limited to, procurement of substitute goods or services;
%% loss of use, data, or profits; or business interruption) however caused and
%% on any theory of liability, whether in contract, strict liability, or tort
%% (including negligence or otherwise) arising in any way out of the use of
%% this software, even if advised of the possibility of such damage.
%% 
%% The views and conclusions contained in the software and documentation are
%% those of the authors and should not be interpreted as representing official
%% policies, either expressed or implied, of Richard Wood.

\documentclass[10pt]{article}

\usepackage{amsmath}
\usepackage{algorithmic}
\usepackage{colonequals}

\newcommand{\bfb}{\mathbf{b}}
\newcommand{\bfx}{\mathbf{x}}

\newcommand{\BigO}{\mathcal{O}}

\title{bdspec---Special-Band-Matrix Solver}
\author{Richard Wood}
\date{\today}

\begin{document}

\maketitle

bdspec solves band matrices, but with a twist.  Instead of the region above the
highest superdiagonal containing zeros, it is filled with values identical to
the highest superdiagonal element of the same row. Such matrices shall be
referred to as special band matrices.

Throughout, \texttt{lb} is the number of subdiagonals, \texttt{ub} is the
number of superdiagonals, and \texttt{n}$\times$\texttt{n} is the dimension of the
special band matrix.

Some examples of special band matrices 
\begin{alignat}{2}
  \nonumber
  &&&\hspace{3.5em}\mbox{\raisebox{-1em}{$\overbrace{\quad}^{\displaystyle\texttt{ub}=1}$}} \\
  &\texttt{n}=6  &
  &\left\{\left(
  \begin{array}{cccccccc}
    \alpha_0 & \beta_0 & \beta_0 & \beta_0 & \beta_0 &\beta_0 \\
            & \alpha_1 & \beta_1 & \beta_1 & \beta_1 &\beta_1 \\
            &         & \alpha_2 & \beta_2 & \beta_2 &\beta_2 \\
            &         &         & \alpha_3 & \beta_3 &\beta_3 \\
            &         &         &         & \alpha_4 &\beta_4 \\
            &         &         &         &         &\alpha_5 \\
  \end{array}
  \right)\right.
  ~,\\
  \nonumber
  &&&\hspace{10.2em}\mbox{\raisebox{2em}{${\displaystyle\texttt{lb}=0}$}}
\end{alignat}
\begin{alignat}{2}
  \nonumber
  &&&\hspace{4.0em}\mbox{\raisebox{-1em}{$\overbrace{\hspace{4em}}^{\displaystyle\texttt{ub}=2}$}} \\
  &\texttt{n}=6  &
  &\left\{\left(
  \begin{array}{cccccccc}
    \beta_0 & \gamma_0 & \delta_0 & \delta_0 & \delta_0 &\delta_0 \\
    \alpha_1 & \beta_1 & \gamma_1 & \delta_1 & \delta_1 &\delta_1 \\
            & \alpha_2 & \beta_2 & \gamma_2 & \delta_2 &\delta_2 \\
            &         &  \alpha_3 & \beta_3 & \gamma_3 &\delta_3 \\
            &         &         & \alpha_4 & \beta_4 &\gamma_4 \\
            &         &         &         & \alpha_5 &\beta_5 \\
  \end{array}
  \right)\right.
  ~.\\
  \nonumber
  &&&\hspace{10.0em}\mbox{\raisebox{2em}{$\underbrace{\hspace{1.0em}}_{\displaystyle\texttt{lb}=1}$}}
\end{alignat}

\section{Storage}

The matrices are stored in compact form to save space. In compact form elements
remain on their original row, but are shifted so that diagonals are stored in
columns and repeated elements are removed.  \texttt{lb} columns of padding are
added to the left to be used for LU factorization.  The compact versions of the
two matrices above are stored in the arrays
\begin{alignat}{2}
  \nonumber
  &&&\hspace{3.4em}\mbox{\raisebox{-1em}{$\overbrace{\quad}^{\displaystyle\texttt{ub}=1}$}} \\
  &\texttt{n}=6  &
  &\left\{\left(
  \begin{array}{cc}
    \alpha_0 & \beta_0 \\
    \alpha_1 & \beta_1 \\
    \alpha_2 & \beta_2 \\
    \alpha_3 & \beta_3 \\
    \alpha_4 & \beta_4 \\
    \alpha_5 & * \\
  \end{array}
  \right)\right.
  ~,
\end{alignat}
\begin{alignat}{2}
  \nonumber
  &&&\hspace{7.8em}\mbox{\raisebox{-1em}{$\overbrace{\hspace{3em}}^{\displaystyle\texttt{ub}=2}$}} \\
  &\texttt{n}=6  &
  &\left\{\left(
  \begin{array}{ccccc}
  *  &        *  & \beta_0 & \gamma_0 & \delta_0 \\
  *  & \alpha_1 & \beta_1 & \gamma_1 & \delta_1 \\
  *  & \alpha_2 & \beta_2 & \gamma_2 & \delta_2 \\
  *  & \alpha_3 & \beta_3 & \gamma_3 & \delta_3 \\
  *  & \alpha_4 & \beta_4 & \gamma_4 & * \\
  *  & \alpha_5 & \beta_5 & *        & * \\
  \end{array}
  \right)\right.
  ~,\\
  \nonumber
  &&&\hspace{3.1em}\mbox{\raisebox{2em}{$\underbrace{\hspace{1em}}_{\displaystyle\texttt{lb}=1}$}}
\end{alignat}
respectively, where $*$ indicates unused elements.

For a special band matrix $A$, to solve the matrix problem $A\bfx =
\bfb$ for vector $\bfx$ Crout LU decomposition with partial pivoting is performed.
The factorization is performed in-place. The first \texttt{lb} columns of the
compact array store the subdiagonals of lower triangular matrix $L$. The
diagonal of matrix $L$ contains ones and is not stored. The remaining
$\texttt{ub}+\texttt{lb}+1$ columns of the compact array store the diagonal and
superdiagonals of the upper triangular matrix U. The rows of the compact array
are re-ordered as pivoting takes place.

A small example helps understand the storage of the LU decomposition.
A special band matrix $A$ with $\texttt{n}=6$, $\texttt{lb}=1$, and $\texttt{ub}=2$:
\begin{align}
A=
  \left(
  \begin{array}{cccccc}
\phantom{+}0.31 & \phantom{+}0.35 & -0.63 & -0.63 & -0.63 & -0.63\\
-0.99 & -0.02 & -0.53 & -0.24 & -0.24 & -0.24\\
      & -0.88 & -0.82 & \phantom{+}0.27 & -0.00 & -0.00\\
      &       & \phantom{+}0.76 & \phantom{+}0.03 & \phantom{+}0.38 & \phantom{+}0.74\\
      &       &       & -0.81 & -0.07 & \phantom{+}0.25\\
      &       &       &       & \phantom{+}0.24 & -0.25\\
  \end{array}
  \right)
  ~.
\end{align}
Compact storage of A:
\begin{align}
  \texttt{bA} = 
  \left(
  \begin{array}{cccccc}
 *  &  *    & \phantom{+}0.31 & \phantom{+}0.35 & -0.63 &\\
 *  & -0.99 & -0.02 & -0.53 & -0.24 &\\
 *  & -0.88 & -0.82 & \phantom{+}0.27 & -0.00 &\\
 *  & \phantom{+}0.76 & \phantom{+}0.03 & \phantom{+}0.38 & \phantom{+}0.74 &\\
 *  & -0.81 & -0.07 & \phantom{+}0.25 &  *  &\\
 *  & \phantom{+}0.24 & -0.25 &    *  &  *  &\\ 
  \end{array}
  \right)
  ~.
\end{align}
The LU decomposition of A:
\begin{align}
L &= \left(
 \begin{array}{cccccc}
\phantom{\phantom{+}}1\phantom{.00}     &       &       &       &       &       \\
      & \phantom{\phantom{+}}1\phantom{.00}     &       &       &       &       \\
-0.31 & -0.39 & \phantom{\phantom{+}}1\phantom{.00}     &       &       &       \\
      &       &       & \phantom{\phantom{+}}1\phantom{.00}     &       &       \\
      &       &       &       & \phantom{\phantom{+}}1\phantom{.00}     &       \\
      &       & -0.68 & \phantom{+}0.47 & -0.28 & \phantom{\phantom{+}}1\phantom{.00}     \\
 \end{array}
 \right)~, \\
U &= \left(
 \begin{array}{cccccc}
-0.99 & -0.02 & -0.53 & -0.24 & -0.24 & -0.24 \\
      & -0.88 & -0.82 & \phantom{+}0.27 & -0.00 & -0.00 \\
      &       & -1.11 & -0.60 & -0.70 & -0.70 \\
      &       &       & -0.81 & -0.07 & \phantom{+}0.25 \\
      &       &       &       & \phantom{+}0.24 & -0.25 \\
      &       &       &       &       & \phantom{+}0.07 \\
 \end{array}
 \right)
 ~.
\end{align}
Compact storage of $L$ and $U$:
\begin{align}
  \texttt{bLU} =
  \left(
  \begin{array}{cccccc}
-0.31 & -0.99 & -0.02 & -0.53 & -0.24 \\
-0.39 & -0.88 & -0.82 & \phantom{+}0.27 & -0.00 \\
-0.68 & -1.11 & -0.60 & -0.70 & -0.70 \\
\phantom{+}0.47 & -0.81 & -0.07 & \phantom{+}0.25 &   *   \\
-0.28 & \phantom{+}0.24 & -0.25 &   *   &   *   \\
  *   & \phantom{+}0.07 &   *   &   *   &   *   \\
  \end{array}
  \right)
\end{align}

The reordering of the rows due to partial pivoting is stored in an integer array \texttt{indx}.
To obtain the permutation matrix $P$ from \texttt{indx}:
\begin{algorithmic}
  \STATE Let $P \colonequals I = $ identity matrix
  \FOR{$j = 0,1,2,\dots \texttt{n}-1$}
  \STATE swap rows $j$ and \texttt{indx}$[j]$ of $P$
  \ENDFOR
\end{algorithmic}
For the above example:
\begin{align}
  \texttt{indx} &= \left(
  \begin{array}{cccccc}
    1 & 2 & 0 & 4 & 5 & 3
  \end{array}
  \right)
  \\
  P &=
  \left(
  \begin{array}{cccccc}
     &1& & & & \\
     & &1& & & \\
    1& & & & & \\
     & & & &1& \\
     & & & & &1 \\
     & & &1& &  \\
  \end{array}
  \right)
  ~.
\end{align}

\section{Functions}
\texttt{void bdspecLUmlt(double *bA, int n, int lb, int ub, double x[], double out[]);}

Calculates $A\bfx$ for special band matrix $A$ stored compactly in array
\texttt{bA}. Results are stored in \texttt{out}.
\vspace{1em}

\texttt{void bdspecLUfactor(double *bA, int n, int lb, int ub, int indx[])}

Performs an Crout $LU$ factorization for special band matrix stored compactly in array
\texttt{bA}. Results are stored in-place as described above.
\vspace{1em}

\texttt{void bdspecLUfactorscale(double *bA, int n, int lb, int ub, int indx[])}

The same as \texttt{bdspecLUfactor}, but partial pivoting is performed based on the size of
elements after being normalized. Normalizing elements involves scaling them by
the element of the largest magnitude that is in the same row.
\vspace{1em}

\texttt{void bdspecLUfactormeschscale(double *bA, int n, int lb, int ub, int indx[])}

The same as \texttt{bdspecLUfactorscale},
but deliberately includes a bug found in the meschach linear algebra library
version 1.2A. The bug is seldom important, and deliberately including it allows careful
testing against Meschach results.
\vspace{1em}

\texttt{void bdspecLUsolve(double *bLU, int n, int lb, int ub, int indx[], double b[])}

Solve $LU\bfx = P\bfb$ for vector $\bfx$. As detailed above, $LU$ is stored
compactly in array \texttt{bLU}, $P$ is stored compactly in \texttt{indx}.
Results are stored in-place in \texttt{b}.

\vspace{1em}

Other included functions in \texttt{bdspec.c} are intended for testing only.
Note that full testing requires the Meschach library to test against.  Self
consistency tests are insufficient because special band matrices are
often nearly singular---for some matrices large (or even infinite) errors are
a fact of life and sometimes they are completely unacceptable.

\section{Complexity}

It should be noted that only \texttt{bdspecLUsolve} has been optimized, other
functions are expected to be used less frequently.

\texttt{bdspecLUmlt} is $\BigO\left(\texttt{n}\times(\texttt{lb+ub})\right)$

\texttt{bdspecLUsolve} is $\BigO\left(\texttt{n}\times(\texttt{lb+ub})\right)$

\texttt{bdspecLUfactor} and its cousins are 
$\BigO\left(\texttt{n}\times\texttt{lb}\times(\texttt{lb+ub})\right)$

\section{Complex versions}

Complex versions of each routine exist:

\texttt{void zbdspecLUmlt(bdspec\_complex *bA, int n, int lb, int ub, bdspec\_complex x[], bdspec\_complex out[]);}

\texttt{void zbdspecLUfactor(bdspec\_complex *bA, int n, int lb, int ub, int indx[])}

\texttt{void zbdspecLUfactorscale(bdspec\_complex *bA, int n, int lb, int ub, int indx[])}

\texttt{void zbdspecLUfactormeschscale(bdspec\_complex *bA, int n, int lb, int ub, int indx[])}

\texttt{void zbdspecLUsolve(bdspec\_complex *bLU, int n, int lb, int ub, int indx[], bdspec\_complex b[])}

\vspace{1em}

The \texttt{bdspec\_complex} type is defined:

\vspace{1em}

\texttt{ typedef struct  $\{$ double r, i; $\}$ bdspec\_complex; }

\vspace{1em}

and is binary compatible with both c++'s \texttt{complex<double>} class and
c99's \texttt{complex double} type.

The complex routines would be slightly neater written in c++ or c99.
Standard c89 code has been used as it is the lowest common denominator.

\section{Bug Reports}

Please send bug reports to \texttt{richbwood@gmail.com}

\end{document}

